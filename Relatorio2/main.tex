\documentclass{article}
\usepackage[a4paper,width=170mm,top=30mm,bottom=30mm]{geometry}
\usepackage[utf8]{inputenc}
\usepackage[T1]{fontenc}
\usepackage[portuguese]{babel}
\usepackage{hyphenat}
\hyphenation{mate-mática recu-perar}
\usepackage{graphicx}
\usepackage{caption}
\usepackage{subcaption}
\usepackage{color}

\title{{\huge Laboratório de Física}\\{\Large Relatório 1}}
\author{Fabio DESTRO 10284667\\Vitor TORRES 10284952\\}
%\date{07 Março 2018}
\date{\today}

\begin{document}
\begin{abstract}
	\indent
	A intenção desta primeira aula prática foi nos familiarizar com conceitos de medidas de grandezas físicas e suas incertezas, levando-nos a trabalhar cálculos com precisão de instrumentos e propagação de erros,
	para assim entender que medições são falhas e tais falhas precisam ser levadas em conta para verificar a confiabilidade de dados de uma certa amostra ou medida, fornecendo a melhor estimativa para um conjuntos de dados.
	Com o intuito de alcançarmos os resultados esperados pela atividade, utilizamos instrumentos de medições de grandezas, sendo eles paquímetro, micrômetro, proveta graduada e balança, junto com cálculos para volumes e derivadas.
	Obtivemos o volume calculado direta e indiretamente do sólido, seu peso e sua densidade, bem como o diâmetro médio de um fio de cobre irregular.
	Através dos resultados obtidos pode-se concluir que a precisão do volume calculado indiretamente é maior do que o calculado diretamente já que o erro encontrado indiretamente é consideravelmente menor. Já com a densidade do sólido, foi possível inferir que a sua composição é alumínio. Em relação ao fio de cobre conclui-se que o desvio padrão das medidas obtidas de seu diâmetro é maior que a imprecisão do micrômetro e portanto o desvio padrão deve ser utilizado como margem de erro para o diâmetro médio do fio.

\end{abstract}
\newpage
\section{Introdução}
\indent

Na primeira prática foi estudado métodos para mensurar tamanho, peso, volume e densidade de objetos, assim como instrumentos capazes de fazer tais medida (paquímetro, micrômetro, proveta graduada e balança), além de observar a ocorrência e a propagação de erros, que são ocasionados pela imprecisão dos instrumentos utilizados para medida.
Como objeto de estudo, foram utilizados um sólido de formato cilíndrico e um fio de cobre e a partir disso, foi calculado o volume do sólido direta e indiretamente, sua densidade e diâmetro do fio.

\section{Metodologia}
\subsection{Primeiro Procedimento}
\indent

Para calcularmos o volume do sólido percorremos dois caminhos distintos, o direto e o indireto:

Diretamente: Utilizando uma proveta graduada com água (precisão de $0.5mm$), aferimos o volume inicial da água, em seguida colocamos o sólido em seu interior e a partir da diferença, temos o volume do sólido.

Indiretamente: Com o auxílio do paquímetro (precisão de $0.05mm$) medimos cada dimensão necessária para o cálculo de seu volume, a saber $V = \pi R^2 h$. Sendo que seu volume é determinado pelo volume de um cilindro grande com a subtração de um menor, como indicado na figura 1.

\begin{figure}[h]
	\centering
	%\includegraphics[scale=0.1]{a0002.png}
	\includegraphics[width=0.5\textwidth]{a0002.png}
	\caption{Representação do sólido a ser estudado}
	\label{fig:solido}
\end{figure}


\subsubsection{Volume Direto do Cilindro}
Dados:

$V_I = 77\;mm\; =$ volume inicial da proveta;

$V_F = 91\;mm\; =$ volume final da proveta;

$\sigma_{V_F} = \sigma_{V_I} = 0.5\;mm\;=$ precisão da proveta.

\[V = V_F - V_I\]
\[V = (91 \pm \sigma_{V_F})-(77 \pm \sigma_{V_I})\]
\[V = 91-77\pm\sigma_{V}\]
\[V= (14 \pm\sigma_{V})\; cm^3\]
\[\sigma_{V}^2 = \left(\frac{dV}{dV_F}\right)^2\sigma_{V_F}^2+\left(\frac{dV}{dV_I}\right)^2\sigma_{V_I}^2\]
\[\sigma_{V} = \pm\sqrt{1^2\cdot0.5^2+1^2\cdot0.5^2}=\pm0.707106781187\]
\[V= (14 \pm0.71)\; cm^3\]


\subsubsection{Volume Indireto do Cilindro}
Dados:

$H = 41.05\;mm\;=$ altura do cilindro maior (externo);

$D = 22.1\;mm\;=$ diâmetro do cilindro maior;

$h = 18.85\;mm\;=$ altura do cilindro menor (interno);

$d = 12\;mm\;=$ diâmetro do cilindro menor;

$\sigma_H = \sigma_D = \sigma_h = \sigma_d = 0.05\;mm\;=$ precisão do paquímetro.

\[V = \pi \left(\left(\frac{D}{2}\right)^2\cdot H-\left(\frac{d}{2}\right)^2\cdot h\right)\]
\[V = \pi \left(\left(\frac{22.1\pm0.05}{2}\right)^2\cdot \left(41.05\pm0.05\right)-\left(\frac{12\pm0.05}{2}\right)^2\cdot \left(18.85\pm0.05\right)\right)\]
\[V = \pi \left(\left(\frac{22.1}{2}\right)^2\cdot 41.05-\left(\frac{12}{2}\right)^2\cdot 18.85\right)+\sigma_V\]
\[V = 13614.74403750607\pm\sigma_V\]
\vline
\[\sigma_V^2=\left(\frac{dV}{dD}\right)^2\cdot\sigma_D^2+\left(\frac{dV}{dH}\right)^2\cdot\sigma_H^2+\left(\frac{dV}{dd}\right)^2\cdot\sigma_d^2+\left(\frac{dV}{dh}\right)^2\cdot\sigma_h^2\]
\[\sigma_V^2=\left(\left(\frac{2\pi DH}{4}\right)^2+\left(\frac{\pi D^2}{4}\right)^2+\left(-\frac{2\pi dh}{4}\right)^2+\left(-\frac{\pi d^2}{4}\right)^2\right)\cdot0.0025\]
\[\sigma_V^2=\left(4D^2H^2+D^4+4d^2h^2+d^4\right)\cdot\frac{0.0025\pi^2}{16}\]
\[\sigma_V=\pm\sqrt{\left(4\cdot\left(22.1\right)^2\cdot\left(41.05\right)^2+\left(22.1\right)^4+4\cdot\left(12\right)^2\cdot\left(18.85\right)^2+\left(12\right)^4\right)\cdot\frac{0.0025\pi^2}{16}}\]
\[\sigma_V=\pm76.10696380652162\:mm^3\]
\[V = (13.61474403750607\pm0.07610696380652162)\:cm^3\]
\[V = (13.61\pm0.08)\:cm^3\]

Comparando o resultado do cálculo direto e indireto do volume do sólido, é possível perceber que os valores encontrados em ambos os métodos são condizentes e estão coerentes, uma vez que os volumes estão próximos e dentro do limite de erro. Além disso, foi possível observar que o método indireto foi mais preciso, uma vez que possui um erro menor, em relação ao método direto.

Isso ocorre já que para as medidas obtidas pelo método indireto foi utilizado um instrumento com maior precisão, paquímetro, em relação à proveta graduada, mesmo com a propagação de erro do cálculo do volume indireto.


\subsection{Segundo Procedimiento}
Dados:

$M = 36.74\;g\; = $ massa do sólido

$\sigma_M = 0.01\;g\; = $ precisão da balança

\[\rho = \frac{M\;(g)}{V\;(cm^3)} = \frac{(36.74\pm0.01)}{(13.61474403750607\pm0.07610696380652162)}\]
\[\rho = 2.698545040493467 \pm \sigma_\rho\]
\[\sigma_\rho^2 = \left(\frac{d\rho}{dM}\right)^2\sigma_M^2+\left(\frac{d\rho}{dV}\right)^2\sigma_V^2\]
%\[\sigma_\rho = \sqrt{\left(\frac{1}{M}\right)^2 0.01 + \left(-\frac{M}{V^2}\right)^2 0.07610696380652162}\]
\[\sigma_\rho =\pm 0.015102845745\]
\[\rho = (2.70 \pm 0.02)\;g/cm^3\]

A partir do cálculo da densidade e com o auxílio de uma tabela com a densidade de diversos materiais foi possível concluir que o material do sólido é o alumínio, já que sua densidade é correspondente já que o alumínio, na tabela de densidades apresentada em sala, tem valor $2.699\;(g/cm^3)$. Com esses dados, comparando grandezas físicas com incertezas, utilizando $|x_1-x_2|<2(\sigma_1+\sigma_2)$ foi possível perceber que são equivalentes.
\subsection{Terceiro Procedimento}

\begin{table}[!ht]
	\centering
	\begin{tabular}{c|c}
		$d\;(mm)$ & $|d_i-\bar{d}|\;(mm)$ \\\hline
		2.83 & 0.06200\\
		2.76 & 0.00800\\
		2.96 & 0.15200\\
		2.73 & 0.03800\\
		2.69 & 0.07800\\
		2.76 & 0.00800\\
		2.83 & 0.06200\\
		2.72 & 0.04800\\
		2.71 & 0.05800\\
		2.73 & 0.03800\\
	\end{tabular}
	\caption{Medidas aferidas e o desvio do padrão}
	\label{tab:my_label}
\end{table}
\[\sum_{i=0}^N \frac{|d_i-\bar{d}|}{N}=\sum_{i=0}^{10} \frac{|d_i-2.768|}{10}=0.0552\]
\indent
Além disso, no cálculo da média entre os diâmetro, fazendo a propagação de erros, é possível encontrar: $\bar{d} = (2.768 \pm 0.01)\;mm$.

Como o desvio padrão foi maior que o erro propagado, deve ser usado o desvio padrão como margem de erro, portanto: $\bar{d} = (2.768 \pm 0.0552)\;mm = (2.77 \pm 0.05)\;mm$

A diferença entre medir o diâmetro sempre no mesmo ponto ou em pontos diferentes é levar em consideração a variação da forma do objeto.

\section{Resultados e discussão}

\indent

Através das medidas diretas e indiretas do volume do objeto, portanto, foi possível observar as variações nos resultados bem como a margem de erro para cada medida. Além disso, através do cálculo da densidade do sólido, foi possível identificar que se trata de um objeto de aluminio. Já sobre a medida do diâmetro do fio de cobre, foi possível identificar certa irregularidade em sua extensão e por isso, para um valor preciso foi necessário utilizar diversas medidas em lugares distintos.

\end{document}

